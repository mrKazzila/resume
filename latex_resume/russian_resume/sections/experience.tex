%====================
% Experience
%====================

\documentclass[a4paper, 12pt]{article}
\usepackage{TLCresume}
\begin{document}

%====================
% EXPERIENCE 0
%====================

\subsection {\href
{https://mindsw.io/}
{Mindsoft}
\hfill 10 2023 - н.в}
\subtext{Python developer \hfill Удаленно}

\begin{zitemize}
    \item Разрабатываю api для работы с Software-defined storage
    \item Пишу юнит и интеграционные тесты
    \item Провожу рефакторинг кода
\end{zitemize}

\textbf {Стэк}: Python, FastApi, Typer, Pydantic, PyTest, CockroachDB, etcd, Gitlab
\vspace*{8pt}

%====================
% EXPERIENCE A
%====================


\subsection {\href
{https://playrix.com/}
{Playrix}
\hfill 06 2020 - 04 2023}
\subtext{QA Engineer \hfill Удаленно}

\vspace*{8pt}
Помимо ручного тестирования занимался автоматизацией используя Python и кастомный фреймворк для автотестов.
\begin{zitemize}
    \item Внедрил систему анализа логов автотестов, что привело к увеличению количества обнаруженных автотестами ошибок на 80\% и сокращению времени на их поиск и исправление
    \item Автоматизировал процессы работы с задачами в Asana, сократив время, затрачиваемое на их обработку. Это позволило экономить до 10\% времени рабочего дня
    \item Улучшил процесс разработки автотестов путем внедрения типизации и использования docstring, что повысило качество кодовой базы и ускорило адаптацию других сотрудников
    \item Разработал и поддерживал более 50 UI-автотестов для проверки основных событий, туториалов и матч-3 в игре
    \item Обеспечил покрытие около 10\% функциональных требований по чек-листам проекта с помощью разработанных автотестов
\end{zitemize}

\textbf {Стэк}: Charles proxy, Python, Git, Asana API, TeamCity, adb, Android Studio


\end{document}
