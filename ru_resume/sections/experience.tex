% Experience
\resumeSubHeadingListStart

% Current job
\resumeSubheading
{Mindsoft}{Октябрь 2023 -- н.в}
{Python developer}{Удаленно}\\
\vspace*{8pt}
\small \textbf {Tech stack}: Python, FastApi, Typer, Pydantic, PyTest, CockroachDB, etcd, Gitlab

\resumeItemListStart
\resumeItem{Разрабатываю api для работы с Software-defined storage}
\resumeItem{Пишу юнит и интеграционные тесты}
\resumeItem{Провожу рефакторинг кода}
\resumeItem{Внедрил и настроил линтер Ruff, что помогло повысить качество и единообразие кода, снижая количество ошибок и времени на ревью}
\resumeItem{Инициировал использование commit convention в команде для повышения читаемости истории изменений и улучшения командной работы с Git}
\resumeItemListEnd

% Prev job
\resumeSubheading
{Playrix}{Июль 2020 -- Апрель 2023}
{QA Engineer}{Удаленно}\\
\vspace*{8pt}
\small \textbf {Tech stack}: Charles proxy, Python, Git, Asana API, TeamCity, adb, Android studio

\resumeItemListStart
\resumeItem{Разработал и внедрил систему анализа логов автотестов, увеличив количество выявленных ошибок на 80\%, что сократило время на диагностику и исправление дефектов}
\resumeItem{Настроил автоматизацию процессов обработки задач в Asana, что позволило сэкономить до 10\% рабочего времени и улучшить управление задачами в команде}
\resumeItem{Внедрил типизацию и docstring для улучшения качества кода автотестов, что ускорило адаптацию новых сотрудников и повысило читаемость кодовой базы}
\resumeItem{Разработал и поддерживал более 50 UI-автотестов для проверки основных событий, туториалов и матч-3 в игре}
\resumeItem{Обеспечил покрытие около 10\% функциональных требований по чек-листам проекта с помощью разработанных автотестов}
\resumeItemListEnd

\resumeSubHeadingListEnd
